\documentclass[12pt]{article}

\usepackage[letterpaper, hmargin=0.75in, vmargin=0.75in]{geometry}
\usepackage{float}
\usepackage{listings}

\setlength{\parskip}{8pt}
\pagestyle{empty}

\title{TrueType Font Program Table Tree Shaking}

\author{
  Eduardo Hauck dos Santos\\
  \texttt{eduardohauck@gmail.com}
  \and
  Patrick Lam\\
  \texttt{p.lam@ece.uwaterloo.ca}
}
\date{}

\lstset{frame=single}

\begin{document}

\maketitle

\section{Description}

This tool is used to reduce the font program table of a
TrueType font by using a tree shaking approach. It can be combined with
subsetting techniques to achieve a better relative reduction from the
original size. Any functions present in the font program table that are
not being called from any other tables in the font during execution will
be eliminated and a new reduced font will be generated as output.

The executable script treeshaking.py can be called passing a
font as a parameter to perform the reduction on the font program table.
Since it has to obtain information about the execution of the font, more
specifically, about the function calls made during execution, we need
to obtain information about its execution. To achieve that, we make use
of the Abstract Interpreter. During abstract execution, values are
abstract data (not actual data) but we can get information about
programs' control flow, including every possible function call made.

\section{How to use}

The executable script treeshaking.py can be called, passing a
font as a parameter, to perform the reduction on the font program table.

\begin{lstlisting}
usage: pyfttreeshake.py fontFile
\end{lstlisting}

\section{Background Information}

%a brief introduction
To be able to perform the tree shaking on the font program table,
we make use of related works focused on TrueType fonts, specifically,
the TTFont class present in the fonttools project \cite{fonttools} 
(an open source project) and the bytecodeContainer and Symbolic Executor
engine, created as part of a thesis project\cite{bytecode} by Wenzhu Man.

%talk about TTFont here
The TTFont class gives us easy access in the TrueType font format,
allowing us to extract data from the font tables, perform analysis on
it, or even manipulating this data and saving it back to the font. 

%talk about Symbolic executor here
The Symbolic Executor engine allows us to to abstractly
interpretate the program without calculating actual values Although
values are abstract,  we can get information about program's control
flow, including which kind of instructions and functions were called.
The abstract execution is not run using the TTFont object itself, but
instead in the bytecodeContainer class, an intermediate class.

%talk about bytecodeContainer here
The bytecodeContainer class, as mentioned, is an intermediate of
the TTFont class. It extracts some of the contents of the a font to make
it more easily manipulable and offers the possibility of translating the
new table values back to a TTFont object. The Symbolic Executor uses a
bytecodeContainer, containing information about TTFont object to execute
tables of a font.

\section{Flow}

The program starts by loading the font file as a TTFont object.
We then create a bytecodeContainer object that encapsulates the main
tables of a Truetype font file: prep, fpgm, cvt and glyf. This object 
is necessary to run the abstract executor.We also create an empty set, 
in which we will insert the labels of functions that were called at some 
point during the execution. This set will be used later on to decide 
which functions we want to keep on the font file.

Whenever a font is loaded, prep is always the first table to run to 
set the environment for glyphs to be loaded. It's important to note that, 
whenever we run a table through the abstract executor, it stores the
resulting context and information about its execution. We are interested 
here specifically on function calls made during the execution.

The first step then is running the prep table through the abstract 
executor. We make a copy of the post-prep environment (we want to be
able to restore this initial state) and update our function label
set with the functions called by prep. We now have everything ready to
execute the remaining tables.

The second step running every glyph through the abstract executor.
This is made on a loop. For each glyph we: 1 - restore the execution
context post-prep, 2 - execute glyph, 3 - update function function call
set with calls made during execution. At the end of it, we have all
functions that can possibly be called by that font.

The third and last step is obtaining the labels of the functions
to be removed (by subtracting the functions called from the list of
functions available on the fpgm table of the font). With them, we can
call the removeFunctions from the bytecodeContainer object - which will
update its function table, removing the ones passed - and finally call
updateTTFont, passing the original TTFont object. It will translate the
contents of bytecodeContainer back to the TTFont object. With that done,
we only need to call the function save from TTFont class to save the
reduced, tree shaken font.

\section{Testing}

To check if the tree shaking technique did not interfere with the
renderization of any glyphs, we developed a script to compare glyphs
from two fonts and check if they are identical. It is interesting to
notice that we are using this script to check if nothing wrong happened
during the tree shaking, but its use can be extended to any sort of
modification applied to the glyphs of a TrueType Font.

The script will compare the two fonts for a given set of glyphs at
five main dpi resolutions: 72x72, 300x300, 600x600, 1200x1200 and
2400x2400dpi. Depending on the number of glyphs to be fetched, and from
where they have to be fetched from (file or URL), the test may take a
while. The default enconding (UTF-8) can be changed via parameter, as
well as the resolution (if a specific one must be tested)

\begin{lstlisting}
usage: pyftcompare.py [options] fontA fontB inputFileOrURL

pyftcompare -- TrueType Glyph Compare Tool

    General options:
    -h Help: print this message
    -v Verbose: be more verbose
    -e CODE Encoding: encoding used to read the input file
    -x HRES Hres: Horizontal resolution dpi
    -y VRES Vres: Vertical resolution dpi
\end{lstlisting}

\clearpage
\begin{thebibliography}{1}

\bibitem{fonttools} The fonttools project. {\em https://github.com/behdad/fonttools } 

\bibitem{bytecode} COI: A First Step towards TrueType Bytecode Analysis.
{\em https://uwspace.uwaterloo.ca/handle/10012/9116 } 

\end{thebibliography}

\end{document}
